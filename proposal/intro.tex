
\chapter{Introduction}

%\remark{
%Things to add: 1) concrete examples such as the GitHub, Amazon, etc. breakdowns
%and other motivations we have in the presentations 2) concrete examples of
%potential benefits (e.g., preview the good stuff from the eval)}
%\remark{
%Introduction does not yet reflect the fact that \sys is the product of extensive
%design decisions made both by experiment/engineering, and by rigorous
%design considering formal reasoning and proofs, one of which we've mechanized.
%Reader should be setup to understand this is a serious design paper. Among expected
%outcomes: reader understands motivation, goals of participatory networking, and
%briefly how it is different from other previous efforts.}

Today's applications, whether running in datacenters, enterprise, campus, or
home networks have an increasingly difficult relationship with the
network. Networks are the shared fabric interconnecting users, applications,
and devices, and fluctuating, unpredictable network performance and reliability
create challenges and uncertainty for network administers, application
developers, and frustrated end-users alike. As a result, software developers,
researchers, and administrators expend considerable effort to work \emph{around}
the network rather than work \emph{with} the network: video conferencing
applications constantly probe network bandwidth~\cite{SkypeTraffic}, overlay networks are used to
re-route traffic~\cite{SkypeTraffic}, network paths are reactively reconfigured based on inferences~\cite{alfares10hedera},
and humans are required to throttle heavy network loads in response to planned
or unplanned shifts in traffic matrices.
%
Using humans for network control, however, is no panacea, having
been responsible for significant recent outages at both
Github~\cite{githubOutage} and
Amazon~\cite{awsOutage}.

At a minimum, packet networks forward data, collect traffic statistics, and
divide traffic based on addresses or other header fields. Increasingly, modern
networks also provide additional services, often implemented via middleboxes,
such as firewalling, compression, encryption, threat-detection, acceleration, and caching.
Yet, all of these features are, for the most part, invisible to the applications
passing traffic through them, or only available via rudimentary interfaces
such as DSCP header bits.

With greater visibility into and control of the network's state,
a conferencing application could request bandwidth for a video call,
and learn via the network that while only a guaranteed audio call is available
now, it could reserve a video call in one hour.
An intrusion detection script on a user machine could request that the
\emph{network} filter traffic from a specific source.
An important RPC service could protect latency-sensitive flows from competing
background traffic.
%An RPC service could inform the network that its flows will be short, but with a
%deadline, which could enable shortest-deadline-first scheduling at
%the switches~\cite{wilson11d3}.
Or, a MapReduce-style application could request
bandwidth guarantees or maximally disjoint paths to improve performance of its shuffle phase.

Such examples suggest that an API should exist between the network's
control-plane and its users, applications, and end-hosts. These
principals need both \emph{read} access to learn the network's
present and potential future conditions, and \emph{write} access to
make independent configuration changes for their own benefit, and
provide helpful knowledge such as future traffic demands, directly
to the network's control-plane.

In this paper, we borrow the concept of \emph{participatory
networks} from \cite{Ferguson:2012a}, in which the network provides
a configuration API to its users, applications, and end-hosts, and
present the design, implementation, and evaluation of the first
practical participatory networking controller for an
OpenFlow-enabled software-defined network.

In the absence of security guarantees and limited authorities,
participatory networks would be places of anarchy. To be usable,
such networks must provide isolation for their independent
principals, preventing malicious users from hogging bandwidth,
dropping packets, or worse. While isolation could be provided
through network virtualization, we believe that is not always the
right abstraction as it hides the fundamentally shared aspect of
networks.  

%{\color{red} By exposing
%conflicts between principals and unhidden information about the
%network state, principals can use this greater awareness to make
%better-informed decisions about how to use or reconfigure the
%network.}

\vskip 0.5em

We call our prototype \sys, COntroller for Participatory nEtworking.
%The \sys controller implements a capability system --
It delegates read and
write authority, with optional restrictions, from the network's
administrators to the users, or applications and hosts acting on
their behalf.  The controller is logically centralized, has a global
view of the network, and implements the principals' high-level
intents by changing the configuration of the actual network devices.
In addition, we implement and evaluate the examples described
above by augmenting four real applications (\xref{sec:Evaluation}).

\sys's user-facing API serves as the next layer on the current SDN stack.
The abstractions provided by software-defined networks allow us
to reason formally about \sys's design, and ensure the network
continues to provide baseline levels of fairness and security, even as principals
dynamically invoke their capabilities.

Our design addresses the two key challenges of participatory networks:
how to safely decompose control and visibility of the network, and how to resolve conflicts
between participants and across requests. 
\sys's solutions to these challenges were developed both through formal reasoning,
and by porting real-world applications to solve existing use cases.
We start with an overview of our solution in \xref{sec:overview},
followed by more in-depth discussions for each challenge in \xref{sec:delegation}
and \xref{sec:conflicts} respectively.


\vskip 0.5em

Many approaches to achieve some of these goals have been previously
proposed including active networking~\cite{ActiveNetworking}, IntServ networking~\cite{rfc1633}, and
distributed reservation protocols such as RSVP~\cite{rfc2205} and NSIS~\cite{rfc5974}.
We discuss their relation to participatory networking in \xref{sec:related-work}.
\sys does not introduce any new
functionalities into the network. Instead, it exposes existing functionalities and
provides a single platform for reasoning about their use. We argue that
this approach provides several advantages: a single target for application
developers, a unified management framework for network administrators,
and, most importantly, the ability to reason across all network resources.

\sys is designed for networks within a single administrative domain
including corporate WANs, datacenters, campus or enterprise
networks, and home networks. In such networks, there is, logically,
a single owner from which authority can be delegated.  \sys's design
does not rely on changes to the end-hosts, such as the use of a
particular hypervisor, making it suitable for networks with
user-owned or managed devices.  

\vskip 0.5em

This paper makes the following contributions:

\begin{enumerate}

%\item {\color{red}We develop a uniform abstraction to enable this ``conversation,'' and design
%a corresponding API.}

\item We implement a fully-functioning SDN controller which allows a
network's administrators to safely delegate their authority using our API.

\item We analyze a previously proposed algorithm for consolidating
hierarchical policies, and propose a new algorithm that reduces the complexity
from exponential to polynomial.

\item We demonstrate our system's usefulness and
practicality on a real OpenFlow testbed using 
microbenchmarks and four real applications enhanced with our API.


\end{enumerate}
