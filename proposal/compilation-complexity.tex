\section{Compiler Complexity}
\label{sec:compiler-complexity}

To realize a policy tree in OpenFlow hardware, we have to compile it
to flow tables for each switch. We use a variation of
Hierarchical Flow Tables (HFT)~\cite{Ferguson:2012b}. A direct
implementation of the HFT algorithm produces flow tables of size
$O(2^n)$, where $n$ is the size of the policy tree. The earlier
algorithm is therefore useless on all but trivial policies.  However,
we make two changes that greatly reduce the complexity:
the modified algorithm yields flow tables
of size $O(n^2)$ in $O(n^2)$ time. This section is an overview of our
results. 

OpenFlow flow tables are simple linear
sequences of patterns and actions. A flow can match several,
overlapping policy atoms in a policy tree and trigger
conflict-resolution that combines their policies. However, in an
OpenFlow flow table, a flow will only trigger the action of the
highest-priority matching pattern.

For example, suppose the policy tree has two atoms with the following
flowgroups:
\[
\begin{array}{l}
\paneshare{X}{Y}{\texttt{tcp}}{\star}{\star} \\
\paneshare{\star}{\star}{\texttt{tcp}}{\star}{\texttt{80}}
\end{array}
\]
Suppose flows that match the first flowgroup -- all flows from $X$ to
$Y$ -- are waypointed through some switch, and that flows that match
the second flowgroup -- all HTTP requests -- are given some bandwidth
reservation.  These two flowgroups overlap, thus a flow may be (1)
waypointed with a reservation, (2) only waypointed, (3) only given a
reservation, or (4) not be affected by the policy.

An OpenFlow flow table that realizes the above two-atom policy tree must have
entries for all four cases.  The original algorithm~\cite{Ferguson:2012b}
generates all possible combinations given trees of size $n$ --- \ie flow tables
of size $O(2^n)$.

We make two changes to prune the generated flow table: (1) we remove
all rules that generate empty patterns and (2) we remove all rules
whose patterns are fully shadowed by higher-priority rules. The
earlier algorithm is recursive, and we prune after each recursive
call.  It is obvious that this simple pruning does not affect the
semantics of flow tables. However, a surprising result is that it
dramatically improves the complexity of the algorithm.

The intuition behind our proof is that for sufficiently large policy trees,
the intersections are guaranteed to produce
duplicate and empty patterns that get pruned. To see this,
note OpenFlow patterns have a bit-vector that determines which fields
are wildcards.  Suppose two patterns have identical wildcard bits and
we calculate their intersection:

First, if the two patterns are identical, then so is their
  intersection. Of these three identical patterns, two get pruned.
Second, if the two patterns are distinct, since their wildcards are
  identical, they exactly match some field differently. Thus, their
  intersection is empty and pruned.

If patterns have $h$ header fields, there are only $2^h$ unique
wildcard bit-vectors. Therefore, if a policy tree has more than $2^h$
policy atoms, it is assured that some intersections create empty or duplicate
patterns that are pruned.

Our full complexity analysis shows that when the
number of policy atoms, $n$, is larger than $2^h$, then the
compilation algorithm runs in $O(n^2)$ time and produces a flow table
of size $O(n^2)$. OpenFlow 1.0 patterns are $12$-tuple, and our current
policies
only use $5$ header fields. Therefore, on policies
with more than $2^5$ policy atoms, the algorithm is quadratic.

\tightparagraph{Updating Flow Tables}

It is not enough for \sys to generate flow tables quickly. It must
also propagate switch updates quickly, as the time required to update
the network affects the effective duration of requests.
The OpenFlow protocol only allows switches to be updated one rule at a
time.  A naive strategy is to first delete all old rules, and then
install new rules. In \sys, we implement a faster strategy: the
controller state stores the rules deployed on each switch; to install
new rules, it calculates a ``diff'' between the new and old
rules. These diffs are typically small, since rule-table updates occur
when a subset of policy atoms are realized or unrealized.


