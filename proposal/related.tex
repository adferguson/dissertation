% !TEX root = proposal.tex


\chapter{Related Work}
\label{sec:related-work}

%\remark{
%{\em Expected Outcome:} Reader understands participatory network's relation to 
%previous works, and believes that it provides several novel contributions. 
%Reviewers have been satisfied that their pet papers have been cited.
%}

%\remark{ {\color{red}
%HotSDN reviewer 2 requested a good comparison between these works and
%FML. ``It would be nice to illustrate why these are important (vs. FML's facilities)
%with well-motivated examples that cannot be easily expressed in FML. (see more
%from review)
%} } 


\tightparagraph{Programming the Network}
%
\sys allows applications and users to influence
network operations, a goal shared by previous research such as
active networking~\cite{ActiveNetworking}. In active networks, principals
develop distributed programs that run in the network nodes. By contrast,
\sys sidesteps active networks' deployment challenges via its implementation
as an SDN controller, their security concerns by providing a much more restricted
interface to the network, and their complexity by providing a logically centralized
view of the network.

\tightparagraph{Using Application-Layer Information}
%
Many previous works describe specific cases in which information
from end-users or applications benefits network configuration,
flexibility, or performance; \sys can be a unifying framework for
these.  For example, Hedera~\cite{alfares10hedera} showed that
dynamically identifying and placing large flows in a datacenter can
improve throughput up to 113\%. \sys avoids Hedera's
inference of flow size by enabling applications and devices to
directly inform the network about flow sizes. 
Wang, \etal~\cite{Wang:2012} propose \emph{application-aware networking},
and argue
that distributed applications such as Hadoop can benefit from
communicating their preferences to the network control-plane, as we
show in \S\ref{sec:ApplicationUsage}.
%
ident++~\cite{naous09ident++}
proposes an architecture in which an OpenFlow controller reactively
queries the endpoints of a new flow to determine whether it should
be admitted. 
TVA is a network architecture in which end-hosts authorize the receipt
of packet flows via capabilities in order to prevent DoS-attacks~\cite{Yang:2005}.
%\treelang also supports this goal, but by allowing end-hosts to safely
%manage network policies for themselves, as we show in \xref{sec:ApplicationUsage}.
By contrast, \sys allows administrators to delegate the
privilege to install restricted network-wide firewall rules, and
users can do so either proactively or reactively (\cf
\S\ref{sec:sshguard}).  

%% RF: there could be a discussion (maybe here, maybe in discussion)
%% saying that there are works that do in-band signaling (Mahout, D3,
%% PDQ), and the limits and benefits of in-band and out-of-band signalling
%% are still up for investigation. Nah, this is for another paper, if we can get
%% a good handle on the question.

%UPnP~\cite{upnp}
%allows applications to control a network gateway, such as to add a
%port-forwarding entry to a NAT table, but is restricted to this setting.
%
Darwin~\cite{Darwin} introduced a method for
applications to request use of a network's resources, including
computation and storage capabilities in network processors. Like
\sys, Darwin accounts for resource use hierarchically. However, it
does not support over-subscription, lacks support for
access control and path management, and requires routers with
support for active networks.
%
Yap, \etal have also advocated for an explicit communication channel
between applications and software-defined networks,
in what they called \emph{software-friendly networks} \cite{Yap:2009}.
This earlier work, however, only supports requests made by a single,
trusted application. By contrast, \sys's approach to delegation,
accounting, and conflict-resolution allow multiple applications to
safely communicate with an SDN controller.



%% [RF: D3 is not really a great example, as we can't operate at the same timescales
%% and this would require a longer discussion.
%D$^3$~\cite{wilson11d3} is a transport protocol for datacenter
%networks that replaces TCP's congestion control with explicit rate
%control to meet transmission deadlines.  End-hosts in D$^3$ request
%sending rates based on a flow's size and deadline, and receive
%explicit rate signals from the routers along the flow's path. 
%%The success of D$^3$ illustrates the benefit of using
%%application-level information to allocate bandwidth throughout the
%%network. 
%We plan to test similar mechanisms in \sys using hints about
%flowgroup properties.



\tightparagraph{Network QoS and Reservations}
Providing a predictable network experience is not a new goal, and there is a
vast body of protocols and literature on this topic.  \sys relies
heavily on existing mechanisms, such as reservations and prioritized
queue management~\cite{Kim:2010,Stoica:1997}, while adding
user-level management and resource arbitration.
\sys also goes beyond QoS, integrating  hints
and guarantees about access control and path selection.
To date, we have focused on mechanisms exposed by OpenFlow
switches; we expect other mechanisms for network QoS %, 
%such as those proposed
%for rate-limiting TCP streams in a distributed fashion~\cite{Raghavan:2007}
could be integrated as well.
%\sys's primary contribution is a unified framework for exposing these 
%mechanisms while delegating authority
%and accounting for resource use within a single network. 

Like \sys, protocols such as RSVP~\cite{rfc2205} and NSIS~\cite{rfc5974} 
provide applications with a way to
reserve network resources on the network. \sys, however, is designed
for single administrative domains, which permits centralized control for policy
decisions and accounting, and sidesteps many of their deployment difficulties. 
\sys provides applications with control over the configuration of network paths, which RSVP and NSIS do not, and goes beyond reservations
with its hints, queries, and access control requests, which can be made instantly or for a future
time. Finally, RSVP limits aggregation
support to multicast sessions, unlike \sys's support for flow groups. % (described in Darwin paper)

%% TODO: RF: this should really relate to signaling mechanisms: in band versus out of band.
%% mahout, d3, pdq, XCP, RCP all use in-band signalling, while we use out-of-band signaling
%% they can react faster, and are more scalable, as may are stateless. We can leverage
%% authentication, use multiple applications, handle conflicts, and, while we have to change
%% applications some times, we don't have to change protocols. (Not very strong argument)
%Modern datacenters provide network quality of service for applications by
%assigning packets to priority queues based on header fields, such as DSCP set
%by applications or end-hosts (\eg ~\cite{curtis11mahout}). \sys provides QoS to untrusted principals by configuring queues reactively, an approach which allows the \sys controller to account for
%resource use and enforce allocation policies.
%%\sys provides a network's operators and end-users with an abstraction layer for
%%delegation and accounting over any QoS implementations which exist in the network.
%By contrast, previous approaches to network QoS, which did not provide central
%accounting, are vulnerable to unscrupulous, or mistaken, end-users who mark all
%packets with the highest priority.


Kim, et al.~\cite{Kim:2010} describe an OpenFlow controller which
automatically configures QoS along flow paths using application-described
requirements and a database of network state. \sys's runtime performs
a similar function for the \priv{Reserve} action, and also supports
additional actions.

Recent works in datacenter networks, such as Oktopus~\cite{Ballani:2011} and
CloudNaaS~\cite{Benson2011cloudnaas}, offer a predictable experience to tenants
willing to fully describe their needs as a virtual network, only admitting those tenants
and networks whose needs can be met through careful placement. This approach
is complementary to \sys's, which allows principals to request resources
from an existing network without requiring complete specification.

%This framework provides
%a single API which end-users and applications to can use to improve
%their network experience through any mechanisms which exist within the
%network. 

%priority classes or TOS bits: yes, they are set by the application, but we are doing
%something like a "request to send?" "clear to send" interaction for the priority bits
%... this lets us do accounting and delegation, e.g. for blocks of time when you can
%set the bits to different levels. one of the problems with previous QOS proposals is
%that people always turn the setting up to the max when they're on their own ...
%we are providing an abstraction layer over the network mechanism which
%implements the QOS and provides it to the operator

\tightparagraph{Software-Defined Networking}
%
\sys is part of a line of research into centralized network management including
Onix~\cite{Koponen:2010}, Tesseract~\cite{Tesseract}, and CoolAid~\cite{CoolAid}.
CoolAid provides high-level requests and intentions about the network's configuration
to its operators; \sys extends this functionality to regular users and applications with
the necessary delegation and accounting, and implements them in an SDN.
\sys builds upon the control-plane abstractions proposed by Onix and Tesseract 
for, respectively, OpenFlow and 4D~\cite{Greenberg:2005} networks.

%\tightparagraph{Software-Defined Networking}
%As discussed previously, the current implementation of \sys was developed as
%an application atop the software stack provided by SDNs.
Recent developments in making SDN practical
(\eg,\cite{Gude:2008,McKeown:2008, Voellmy:2011}) greatly improve
the deployability of \sys.  
Resonance~\cite{Nayak:2009} delegates access control to an automated
monitoring system, using OpenFlow to enforce policy decisions.
Resonance could be adapted to use \sys as the mechanism for taking
action on the network, or could be composed with \sys using a
library such as Frenetic~\cite{Foster:2010}.

Expressing policies in a hierarchy is a natural and common way to represent
delegation of authority and support distributed authorship. Cinder~\cite{Roy:2011}, for example, uses a hierarchy of \emph{taps} to provide isolation, delegation, and division of the right to consume
a mobile device's energy.
\sys uses HFTs~\cite{Ferguson:2012b} as a natural way to express, store, and
manipulate these policies
directly, and still enable an efficient, equivalent linear representation of the policy.

FlowVisor~\cite{Sherwood:2010} divides a single network into multiple slices
independently controlled by separate OpenFlow controllers. FlowVisor
supports delegation -- a controller can re-slice its slice of the
network. Each of these controllers sends and receives primitive
OpenFlow messages. In contrast, \sys allows policy authors to
state high-level, declarative policies with flexible conflict resolution.


%The \sys controller
%is built on Nettle~\cite{Voellmy:2011}, a platform for writing OpenFlow
%controllers in a functional reactive style, and allows policy authors to write
%higher-level policies than those Nettle supports natively.



\tightparagraph{Networking and Declarative Languages}
\sys's design is inspired by projects such as the Margrave tool for firewall 
analysis~\cite{Nelson:2010} and the Router Configuration
Checker~\cite{Feamster:2005}, which apply declarative languages to
network configuration. Both use a high-level language to detect configuration 
mistakes in network policies by checking against predefined constraints. 
\sys, however, directly integrates such logic into the network
controller.
%NDlog~\cite{Loo:2009} adapted the declarative Datalog language to ease the 
%programming of distributed applications and protocols; \sys receives
%requests from distributed clients, but their evaluation is centralized.

FML~\cite{Hinrichs:2009} is a Datalog-inspired language for writing
policies that also supports distributed authorship.  
The actions in \sys are inspired by
FML, which it
extends by involving end-users, adding queries and hints, and
introducing a time dimension to action requests.
In an FML policy,
conflicts are resolved by a fixed scheme -- deny overrides waypoints,
and waypoints override allow. By contrast, \sys offers more
flexible conflict resolution operators.
%For example, within a single \treelang policy tree,
%one policy node may specify ``allow overrides deny,'' while another
%specifies ``deny overrides allow.''
%
FML also allows policies to be prioritized in a linear sequence (a
\emph{policy cascade}). \sys can also express a prioritized sequence of
policies, in addition to more general hierarchies.  
%For example, \sys
%uses an inverted ``child overrides parent'' conflict-resolution scheme
%(\xref{sec:conflict-resolution-operators}) by default, and the author of an individual
%policy node can adopt a more restrictive ``parent overrides child''
%scheme. FML does not support both ``child overrides parent'' and
%``parent overrides child'' schemes simultaneously.


{\color{red}
%Sharing data center network \cite{S hieh:2011, Popa:2011}.
%Predictable datacenter networks \cite{Ballani:2011}.

%Do we want to say anything about how we compare to the IETF 82 BOF session on 
%``Software Driven Networks''?
%\url{https://www.ietf.org/proceedings/82/minutes/sdn.html}
%\url{http://lucidvision.com/pipermail/sdnp/}
}

%\treelang gives policy writers a programming model in which a central
%controller sees all packets. This model is inspired by Frenetic~\cite{Foster:2010}, which uses 
%NetCore~\cite{Monsanto:2012} as its
%language for expressing forwarding policies. 
%In contrast to NetCore,
%\treelang supports \emph{hierarchical policies}, which naturally support
%distributed authorship. A key element of \treelang is allowing
%overlapping and conflicting policies to co-exist in a policy tree,
%as it resolves conflicts with arbitrary, user-defined
%operators.
%
%NetCore allows policy writers to match packets with arbitrary
%predicates; when predicates are not realizable on switches, packets
%are sent to the NetCore controller, which uses \emph{reactive
%  specialization} to install exact-match rules. In contrast,
%\treelang's matching rules are simpler and realizable on OpenFlow
%switches, though we do not anticipate any problem supporting reactive
%specialization.


% see: http://books.google.com/books?id=EMiBsbiYA-kC&pg=PA31&lpg=PA31
The eXtensible Access Control Markup Language (XACML) provides
four combiner functions to resolve conflicts between subpolicies~\cite{gm:xacml-1.1}.
These functions are designed for access control decisions and assume an ordering
over the subpolicies. % eg, ``first-applicable''
By contrast, HFTs support user-supplied operators designed for several actions
and consider all children equal.


