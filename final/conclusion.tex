\chapter{Conclusion}
\label{sec:conclusion}

\section{Participatory Networking}
\label{sec:exodus-conclusion}

The design and configuration of today's networks is already informed
by application needs (\eg,
networks with full-bisection bandwidth for MapReduce-type frameworks,
or deadline-based queuing~\cite{Ballani:2011} for interactive web services).
\sys provides a way for the network to solicit and react to such needs
automatically, dynamically, and at a finer timescale than with human input.
To do this, our design overcomes the two challenges of decomposing network
control, and resolving conflicts between users' needs.

%Reiterate how \sys is about exposing network capabilities and information
%to the end-users. We believe that it can be used as a building-block for other
%proposals, and it could be integrated alongside other applications in an SDN
%controller (eg, load-balancing, Hotel/Campus authentication portals, firewalls
%and other security applications such as the HoneyNet work, etc.)

\section{Exodus}
\label{sec:exodus-conclusion}

Exodus is the first SDN migration tool which directly migrates existing network
policies to equivalent SDN controller software and an OpenFlow-based
network configuration. Automatic migration allows network operators familiar
with their own networks, but not SDN, to quickly explore the benefits of this new
approach.
By generating code in a high-level, rule-based language, Exodus
makes it easy to bootstrap a new network controller which can evolve
at the frenetic pace of enterprise network environments~\cite{kim11evolution}.
The high-level semantics of the generated program opens the avenue
for change-impact analysis, and potential refactoring of the physical 
configuration of the network, bringing the full benefits of an SDN
deployment.
No matter the migration strategy eventually employed, Exodus gives
network administrators a concrete, working prototype from which to begin
discussion and compare solutions.

\section{Lessons from Building SDN Controllers}
\label{sec:building-controllers}