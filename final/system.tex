% !TEX root = ferguson-dissertation.tex
\chapter{The \sys Controller}
\label{sec:FullSystem}

%\remark{
%{\em Expected Outcome:} Reader understands the smaller components of the system 
%and how the share tree and the policy trees fit together with them.
%}

%\remark{
%Need to say we got the policy tree implementation and compiler described
%earlier, along with several other components that are in the figure, but not
%detailed in this paper.
%}

The complete \sys system integrates the previously described
components into a fully-functioning SDN controller, as depicted in
\autoref{f:system}.  It manages simultaneous connections with the
network's principals and its switches. In this role, it is
responsible for implementing both our participatory networking API,
as well as the details of computing default forwarding routes,
transmitting OpenFlow messages, and reacting to network changes such
as switches joining and links failing.  To accomplish these tasks,
the \sys controller maintains three data structures: the share tree,
a sequence of policy trees, and a \emph{network information base}
(NIB), described below.

We have developed a prototype \sys controller using Haskell and the
Nettle library for OpenFlow~\cite{Voellmy:2011}. Although
we chose OpenFlow as our substrate for implementing \sys, participatory networking's design
does not depend on OpenFlow. \sys could be implemented using other
mechanisms to control the network, such as 4D~\cite{Greenberg:2005},
MPLS, or a collection of middleboxes.

A prototype release is available on Github, and we provide a virtual
machine for Mininet-based evaluation on our website.\footnote{\url{http://pane.cs.brown.edu}.} The release also includes a Java library
which implements an object-oriented interface to \sys's text API.

The \sys controller is an entirely event-driven multicore program.
The three primary event types are incoming \sys API messages,
incoming OpenFlow messages, and timer events triggered by the start
or finish of previously accepted requests or realizable hints.

API messages always specify a share on which they
are operating. When a message arrives, the \sys controller first
uses the share tree to determine whether it is authorized, and then,
for requests, whether it is feasible by consulting the policy
trees, as described in the previous chapters.

When requests start and expire, the \sys controller compiles the new
policy tree to a set of switch flow tables, translating
high-level actions to low-level operations on individual
switches in the network. For example, a $\priv{Reserve}(n)$ action
becomes a circuit of switch queues and forwarding rules that direct
packets to those queues.  As we will describe next, \sys's runtime
uses its NIB and a default forwarding algorithm to realize this and
other actions.  Our implementation constructs a spanning tree and
implements MAC learning as its forwarding algorithm.

When possible, \sys uses the slicing extensions to OpenFlow 1.0 to
create and configure queues, and out-of-band commands when necessary.  While
OpenFlow allows us to set expiry timeouts on flow table entries, our
controller must explicitly delete queues when reservations expire.

As Reitblatt, et al.~\cite{reitblatt12consistent} articulate, it is
a challenge to update switch configurations without introducing
inconsistent, intermediate stages. Our implementation does not
presently address this issue; we anticipate confronting this
problem in the future.

\section{\sys's Network Information Base}
\label{sec:NIB}

A network information base (NIB) is a database of network elements
-- hosts, switches, ports, queues, and links -- and their
capabilities (\eg, rate-limiters or per-port output queues on a
switch). The runtime uses the NIB to translate logical actions to a
physical configuration, determine a spanning tree for default packet
forwarding, and to hold switch information such as manufacturer,
version, and its ports' speeds, configurations, and statistics.

For example, \sys's runtime implements a bandwidth reservation,
$(M,\priv{Reserve}(n))$, by querying the NIB for the shortest path
with available queues between the corresponding hosts.  Along this
path, \sys creates queues which guarantee bandwidth $n$, and flow
table rules to direct packets matching $M$ to those queues. We chose
this greedy approach to reserving bandwidth for simplicity, and
leave the implementation of alternatives as future work.

\sys also uses the NIB to install $\textbf{Deny}$ rules as close as
possible to the traffic source. If the source is outside our
network, this is the network's gateway switch. If the source is
inside the network, packets are dropped at the closest switch(es)
with available rule space.

The NIB we implement is inspired by Onix~\cite{Koponen:2010}. It
uses a simple discovery protocol to find links between switches, and
information from our forwarding algorithm, such as ARP requests, to
discover the locations of hosts. 

\section{Additional Features}

The \sys runtime supports several additional features beyond the
requests, hints, and queries previously described. Principals are
able to query \sys to determine their available capabilities,
examine the schedule of bandwidth availability, create
sub-shares, and grant privileges to other principals.  \sys's API
also provides commands to determine which existing requests and
shares can affect a specified flowgroup; this is particularly useful
for debugging the network, such as to determine why certain traffic
is being denied. 

Beyond the API, the \sys controller also includes an administrative
interface which displays the current state and configuration of the
network, real-time information about the controller's performance
such as memory and CPU usage, and allows the dynamic adjustment of
logging verbosity.

\section{Fault Tolerance and Resilience}
\label{sec:dynamics}

While the principals in \sys are
authenticated, they do not need to be trusted as privileges are
restricted by the system's semantics. We recognize, however, that it
may be possible to exploit combinations of privileges in an untoward
fashion, and leave such prevention as future work.

Our prototype implementation of \sys is currently defenseless against
principals which issue excessive requests; we leave such protection against
denial-of-service as future work, and expect \sys's requirement for
authenticated principals to enable such protections.

The \sys controller must consider two types of failures.  The first is
failure of network elements, such as switches or links, and the
second is failure of the controller itself.

When a switch or link fails, or when a link's configuration changes,
the \sys runtime must recompile the policy tree to new individual
switch flow tables, as previously used paths may no longer be
available or acceptable. Because the underlying network has changed,
this recompilation step is not guaranteed to succeed.  If this
happens, we defer to \sys's first come-first serve service model,
greedily replaying requests to build a new policy tree which
does compile; implementing this simply requires annotating the
current policy tree's policy atoms with the order in which they were
created.  Developing a more sophisticated approach to
re-constructing a feasible policy tree, perhaps taking
advantage of priorities, or with the goal of maximizing the number
of restored requests, remains as future work. 

To handle failure of the controller, we can keep a database-like
persistent redo log of accepted requests, periodically compacted
by removing those which have expired. Upon recovery, the
\sys controller could restore its state from this log.  In production
settings, we expect the \sys controller to be deployed on multiple
servers with shared, distributed state. Switches would maintain
connections to each of the controllers as newer OpenFlow
specifications support. We leave the design and analysis of both
options as future work.  Because network principals use \sys in an
opt-in fashion to receive predictable performance, a complete
runtime failure would simply return the network to its current state
of providing best-effort performance only.
