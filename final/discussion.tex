\chapter{Discussion and Future Work}
\label{sec:discussion}

%{\color{red}
%AF:Rodrigo: Our relation with the laissez-faire nature of today's networks and the
%end-to-end principle.?
%RF:Nothing to declare now, I think we should skip the philosophical discussions for now...
%}

%% Integrate with other Frenetic components %%

The \sys prototype implements a complete OpenFlow controller: it
determines network-wide policies by realizing requests and hints,
monitors switches to respond to queries, determines routing, and
updates switches as the policy changes. We've built each of
these components as separate modules, and compose them using
composition abstractions derived from
Frenetic~\cite{Foster:2010}. Frenetic allows SDN controllers to be
built from independent modules. Therefore, it would be straightforward
to compose \sys with other Frenetic modules.

%% Trees vs. DAGs %%

In the \sys prototype, policies are arranged as trees. This is
not a fundamental design decision -- trees make it easy to implement
resource accounting and trace the provenance of delegation decisions. However,
generalizations such as DAGs should be possible.

An interesting generalization of \sys that we plan to implement is a
more flexible language for requests. A principal should be able to
describe the resources they need as simple constraints. For example,
``reserve 5 Gbps for 1 hour within the next 5 hours.'' Such requests
would give the \sys controller more scheduling flexibility. They would
thus improve network performance and lead to greater accepted requests.

%%  New kinds of requests and devices %%

We are also investigating allowing principals to request other kinds
of resources, \eg latency and jitter. Current OpenFlow implementations
lack support for controlling these traffic properties. In the
future, we hope to borrow techniques from works such as Borrowed
Virtual Time~\cite{BVT} or Hierarchical Fair Service
Curves~\cite{Stoica:1997}, which integrate and balance the needs of
throughput- and latency-sensitive processes.

\tightparagraph{\sys as a building block} We believe the primitives
provided by \sys could serve as an implementation substrate for a
variety of recent proposals in the networking literature.

For example, Coflow~\cite{Chowdhury12coflow} proposes an abstraction which
groups several related flows, allowing the network control-plane to better
optimize an application's communication. By adding support for transactions, we
expect \sys could be used to implement Coflow -- each coflow would map
to a sequence of \sys requests grouped by a transaction.

FairCloud~\cite{Popa:2012}, which develops several policies for fairly
dividing datacenter bandwidth, should be implementable using \sys's
reservation and rate-limit requests.

Finally, because \sys allows applications to reserve guaranteed
bandwidth, such applications could skip TCP's slow start phase, or
even allow for the network control-plane to be involved in setting
congestion control and other parameters, as in proposals such as
XCP~\cite{Katabi:2002} and OpenTCP~\cite{OpenTCP}.

\sys could also integrate better support for middlebox-based services,
perhaps by integrating the approach advocated for by Gember,
\etal~\cite{Gember:2012}.
