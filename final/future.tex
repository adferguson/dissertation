% !TEX root = ferguson-proposal.tex

%While the principals in \sys are
%authenticated, they do not need to be trusted as privileges are
%restricted by the system's semantics. We recognize, however, that it
%may be possible to exploit combinations of privileges in an untoward
%fashion, and leave such investigation and prevention as future work.

\chapter{Proposed Work}
\label{sec:proposed}

A key feature of \sys is its \emph{participatory} nature -- it enables
users to participate in the management of the network and directly affect its 
configuration without involving a fully-trusted administrator.
While the \sys controller authenticates users and can log operations for
audit pruposes, users may not always behave appropriately, either willfully 
or accidentally. Because there are multiple independent parties directing the 
configuration of the network, we would like to prevent ``bad''
configurations automatically. ``Bad'' configurations can take many forms:

{\singlespacing
\begin{enumerate}
\item Configurations with routing loops 
\item Configurations with blackholes (\ie, routes which cause packets to be dropped)
\item Routes which violate isolation guarantees, which must be met for practical 
(eg, ``don't let customer traffic on the administrative network''), contractual 
(eg, ``don't allow Pepsi to see Coke's traffic''), or legal reasons (eg, ``don't 
allow medical traffic to reach the wider internet'')
\item Confusing configurations, in which rules are ``shadowed'' by other rules, thus 
obscuring the resulting actions, or which have superfluous rules, which will 
never have an effect.
\end{enumerate}
}

Before continuing, we wish to note that there are two distinct questions here. 
The first is whether making a proposed change causes the network configuration to 
become ``bad,'' and the second is whether granting a proposed privilege
(a capability) to make a certain set of changes \emph{can} lead to bad
configurations in the future. Considering \sys as a capability-based system, 
the first is about the exercise of existing capabilities, and the second is 
about granting new capabilities. Or, in the language of \sys itself, the first 
is about modifications to the PolicyTree, and the second is about modifications 
to the ShareTree.

The recently developed Header Space Analysis (HSA)~\cite{HSA} and
VeriFlow~\cite{VeriFlow} provide tools to detect (and optionally prevent) many bad
configurations from becoming instantiated in the network. These techniques are
independent of the controller platform, and intercept and analyze update traffic
destined to the switches from the controller. By running HSA or VeriFlow as part
of the \sys controller, requests which create bad configurations could be prevented.

However, HSA and VeriFlow cannot help users reason about \emph{potential}
configurations directly. Potential configurations include the space of configurations
made possible by granting a new privilege in \sys, newly compiled configurations
triggered by adding or removing physical elements in the network, and new
policies introduced by modifying the SDN controller (\eg, by loading a new module
or application). Reasoning about these potential configurations is a form
of \emph{change-impact analysis}, and PANE's current conception of the
\emph{strict} keyword is a primitive version of  this -- a
``set and check for expected result''-type operation.

These tools are also important for letting an administrator run ``what-if'' scenarios, 
or ask questions about the existing policy. They are helpful, for example, when 
a new administrator has joined an organization and needs to understand the 
rules created by a previous administrator. For example:

{\singlespacing
\begin{enumerate}
\item is everything configured for service X to work? 
\item can faculty members work with service X? if not, what do we need to adjust?
\item can I block any host in the network? if not, which hosts can I block?
\item who can block any host in the network? or who can block any English 
department host?
\end{enumerate}
}

These last questions are about reachability analysis on the capability graph.

Analyzing these questions in \sys can be complicated because some shares 
may apply to a user's traffic, while other shares may apply to an application's 
traffic: How do we block Alice from exercising a privilege which she can get using 
a share for application Foo? What is the appropriate place in the ShareTree to install such 
a block? This block might be viewed as an \emph{invariant} on the ShareTree which we 
would like to maintain if, for example, the share granted to application Foo 
was created \emph{after} the block on Alice was created. In this simple example, 
as long as a ``deny overrides allow'' policy operator was in the right part of 
the tree where these two policy atoms met, we would be fine -- but what if the 
policy operator changes? We should be able to alert the administrator to the 
effect of this change.

A second ShareTree invariant we are interested in supporting is one to
maintain a level of baseline fairness between users. More concretely, can
we prevent a misconfiguration of the token buckets, such that one user could
monopolize the reservations on a link? Or that some user may never be able
to reserve bandwidth?

Finally, because SDN controllers provide programmers and administrators
with an abstraction of the network, we are interested in the fragility of this
abstraction. What would be the impact of removing a switch or a link? Is there
sufficient redundancy to support critical flows, such as telepresence feeds
for remote surgery, during these failures?

By combining an SDN controller's global knowledge with the structured
information provided by \sys's end-users, our proposed work seeks to give
administrators the necessary answers to delegate their network's management
with confidence.

\begin{comment}
Several previous works have explored how to prevent these configurations in 
non-participatory networks, including:
* Margrave: 
http://www.cs.brown.edu/~sk/Publications/Papers/Published/nbfdk-margrave-firewal
l/paper.pdf
* Header Space Analysis: 
https://www.usenix.org/system/files/conference/nsdi12/nsdi12-final8.pdf
* Feamster's RCC: http://nms.csail.mit.edu/papers/rcc-nsdi-camera.pdf

and could be applied to detecting bad configurations at the PolicyTree-level. 
When applied in an online fashion (that is, when a new rule is considered for 
insertion), these tools provide a type of "change-impact analysis" -- that is, 
does the impact of the change I just made agree with my intent? The key 
challenge here is to capture the user's intent in some reasonable fashion.

It's about 
running a tiny version of that impact analysis tool in the frontend to 
determine if we should accept the proposed change.  One of the cool things 
about `strict` is that it helps when there are many concurrent updates 
happening. If I read the state of the PolicyTree, decide on a rule, and set it 
with `strict`, I know that it will apply exactly as I intended. However, if 
there are have been updates which would have affected it, then the change is 
rejected; this is why `partial` may better because it will be applied either 
way.
\end{comment}
