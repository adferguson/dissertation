% !TEX root = ferguson-dissertation.tex
\chapter{Introduction}

{\bf Thesis Statement} Modern applications can benefit from read and write interaction with a
network's control-plane, yet such interaction is currently unsupported due to
security and fairness concerns. We design a practical and feasible platform for
software-defined networks (SDNs) that addresses these challenges through policy
delegation, and demonstrate its benefits. Furthermore, we can assist
administrators migrating to these new networks by directly translating their
existing network configurations to SDN controller software.

\vskip 1em

\section{The Need for Delegated Network Management}

Today's applications, whether running in datacenters, enterprise, campus, or
home networks have an increasingly difficult relationship with the
network. Networks are the shared fabric interconnecting users, applications,
and devices, and fluctuating, unpredictable network performance and reliability
create challenges and uncertainty for network administrators, application
developers, and frustrated end-users alike. As a result, software developers,
researchers, and administrators expend considerable effort to work \emph{around}
the network rather than work \emph{with} the network: video conferencing
applications constantly probe network bandwidth~\cite{SkypeTraffic}, overlay networks are used to
re-route traffic~\cite{SkypeTraffic}, network paths are reactively reconfigured based on inferences~\cite{alfares10hedera},
and humans are required to throttle heavy network loads in response to planned
or unplanned shifts in traffic matrices.
%
Using humans for network control, however, is no panacea, having
been responsible for significant recent outages at both
Github~\cite{githubOutage} and
Amazon~\cite{awsOutage}.

At a minimum, packet networks forward data, collect traffic statistics, and
divide traffic based on addresses or other header fields. Increasingly, modern
networks also provide additional services, often implemented via middleboxes,
such as firewalling, compression, encryption, threat-detection, acceleration, and caching.
Yet, all of these features are, for the most part, invisible to the applications
passing traffic through them, or only available via rudimentary interfaces
such as DSCP header bits.

With greater visibility into and control of the network's state,
a conferencing application could request bandwidth for a video call,
and learn via the network that, while only a guaranteed audio call is available
now, it could reserve a video call in one hour.
An intrusion detection script on a user machine could request that the
\emph{network} filter traffic from a specific source.
An important RPC service could protect latency-sensitive flows from competing
background traffic.
%An RPC service could inform the network that its flows will be short, but with a
%deadline, which could enable shortest-deadline-first scheduling at
%the switches~\cite{wilson11d3}.
Or, a MapReduce-style application could request
bandwidth guarantees, or maximally disjoint paths, to improve performance of its shuffle phase.

Such examples suggest that an API should exist between the network's
control-plane and its users, applications, and end-hosts. These
principals need both \emph{read} access, to learn the network's
present and expected future conditions, and \emph{write} access, to
make independent configuration changes for their own benefit, and
provide helpful knowledge, such as future traffic demands, directly
to the network's control-plane.

In this thesis, we introduce the concept of \emph{participatory
networking}, in which the network provides
a configuration API to its users, applications, and end-hosts, and
present the design, implementation, and evaluation of the first
practical participatory networking controller for an
OpenFlow-enabled software-defined network.

In the absence of security guarantees and limited authorities,
participatory networks would be places of anarchy. To be usable,
such networks must provide isolation for their independent
principals, preventing malicious users from hogging bandwidth,
dropping packets, or worse. While isolation could be provided
through network virtualization, we believe that is not always the
right abstraction, as it hides the fundamentally shared aspect of
networks.  

% new
By contrast, participatory networks reveal the conflicts between
their principals, and directly expose the control-plane's view of
the network. Principals can use this greater awareness to make
better-informed decisions about how to use or reconfigure the
network.

% new

\section{Software-Defined Networking}

The recent development of Software-Defined Networks (SDNs)
%which provide a logically centralized abstraction of network
%control,
%built on programmable devices,
offers a platform to realize the vision of participatory networks~\cite{Greenberg:2005,McKeown:2008}.
SDNs separate
the logic that controls the network from its physical
devices, allowing configuration \emph{programs} to operate on a
high-level, global, and consistent view of the network. 
%  allowing network-wide policies to be determined
% centrally~\cite{Casado:2009, Gude:2008}, and expressed at a higher
% level of abstraction than configuration of individual network
% elements~\cite{Foster:2010,Hinrichs:2009, Kim:2010}
This change has brought significant advances to datacenter and
enterprise networks, such as high-level
specification of access control~\cite{Nayak:2009} and
QoS~\cite{Kim:2010}, safe experimentation~\cite{Sherwood:2010},
in-network load-balancing~\cite{wang11wild}, and seamless VM 
migration~\cite{Erickson:2008}.

In a traditional network, the control-plane, which determines how
to transmit packets, is distributed across the network's switches. Each
switch independently runs distributed algorithms (\eg, Bellman-Ford) to determine the
packet forwarding policy. By contrast, an SDN has a logically centralized
controller, running on a commodity server platform, which calculates
a forwarding policy (\eg, with Dijkstra's algorithm). The controller implements
this policy by programming the switches' data-planes, which process
traffic at line-rate.

This programming is accomplished via a protocol such as
OpenFlow~\cite{McKeown:2008}, which provides a common
abstraction of a switch's data-plane. The principal component of this
abstraction is a sequence of prioritized ``Match-Action'' pairs, where
matches list a set of packet header fields, and actions specify handling
of the matching packets: transmit via a port, place in a queue, drop, etc.

\section{Migrating to Software-Defined Networks}

Software-defined networks also offer to simplify the management
of enterprise networks, a notoriously challenging problem~\cite{
benson09unraveling,Casado07Ethane, casado06sane, kim11evolution,sung11systematic,
  yu11vpn}. 
SDNs ease the evolvability of the network by centralizing configuration
and management, and enable the use of modern programming
languages and verification techniques.

However, migrating from an existing, working network environment to
an SDN presents a formidable hurdle~\cite{jain++:sigcomm13-google-sdn}.
Enterprises and administrators are familiar
with, and quite likely depend upon, the specific behavior of existing
configurations, and any SDN replacement will need to begin with identical
behavior.

Unfortunately, these networks can be large 
and complex~\cite{kim11evolution,levin13panopticonTR,zeng12test}, with network behavior
defined by myriad policies, 
%both explicit and implicit, 
usually specified for each individual device, in a variety of
languages that configure distributed programs embedded in these
devices. 
The scale and complexity of the aggregate
behavior of these rules means the process of creating a
controller program for an equivalent SDN is non-trivial.
For example, Purdue's network was reported to
have over 15,000 hosts, 200 routers, 1,300 switches, and 182
VLANs in 2011~\cite{sung11systematic}. 
%
Stanford's backbone, whose publicly available configuration we use in
our evaluation (\Cref{sec:evaluation})~\cite{zeng12test}, has two
large border routers connected through 10 switches to 14 Operational Zone
routers. In aggregate, the configuration comprises 757,000 forwarding entries
and 1,500 ACL rules. 


Today, there are a few options for incrementally migrating a physical
network to an SDN~\cite{Casado07Ethane, Casado:hotsdn2012-fabric, levin13panopticonTR}, 
which we review later in the dissertation. However, a common deficiency
of these migration paths is the need to rewrite network policies
and configurations from scratch, a tall order for busy network operators~\cite{Barrett04fieldstudies}.
For a risk-averse network administrator,
this may be reason enough not to migrate. 


%However, few of the world's enterprises are currently using
%SDNs. Instead, most network behavior is defined by the concurrent
%execution of distributed programs running on distinct devices; these
%programs are written in a variety of vendor-specific special-purpose
%languages. Because enterprises and administrators are familiar
%with, and quite likely depend upon, the specific behavior of these
%programs, any SDN replacement will need to begin with identical
%behavior. Unfortunately, 
%In~\cite{levin13panopticonTR} the authors
%evaluate their design on a local campus network comprising 1713 L2 and L3
%switches. Finally, in~\cite{kim11evolution}, the authors report that the
%Georgia Tech and Wisconsin campuses comprise respectively 1097 and 1624
%routers, firewalls, and switches. 

%
This dissertation directly addresses the problem of migrating distributed network
configurations to equivalent SDN controller programs, and presents Exodus, a
system we developed for performing this conversion.
The controllers Exodus generates are written in Flowlog~\cite{Nelson:2014flowlog},
a language for SDN programming we designed out of our experience building
the PANE controller.

The development of PANE, Flowlog, and Exodus has exposed a number of
deficiencies in the current state of SDN development.
We close with a discussion of these deficiencies relative to traditional networking
(\Cref{sec:tradeoffs}), and within the modern software-stack itself (\Cref{sec:conclusion}).

\begin{comment}

\vskip 0.5em

In summary, this thesis makes the following contributions:

\begin{enumerate}

%\item {\color{red}We develop a uniform abstraction to enable this ``conversation,'' and design
%a corresponding API.}

\item We implement a fully-functioning SDN controller which allows a
network's administrators to safely delegate their authority using our API.

\item We analyze a previously proposed algorithm for consolidating
hierarchical policies, and propose a new algorithm that reduces the complexity
from exponential to polynomial.

\item We demonstrate our system's usefulness and
practicality on a real OpenFlow testbed using 
microbenchmarks and four real applications enhanced with our API.


\end{enumerate}

\end{comment}

%\section{Proposed Work}
%
%Participatory networks enable users to directly specify their intentions to
%the network's control-plane. To date, we have shown this interaction
%can occur without compromising the network's security or fairness by
%safely decomposing control and visibility of the network, and resolving 
%conflicts in a principled and intuitive manner
%(Chapters \ref{sec:overview}-\ref{sec:FullSystem}).
%Furthermore, we have shown that such interaction
%can improve user experience (Chapter \ref{sec:Evaluation}).
%In our proposed work (Chapter \ref{sec:proposed}), we plan to reason further \emph{about} these intentions:
%is the network configured to support its users' intentions? if so, how well?
%which intentions can be realized? how will one user's intentions affect another's?
%We hope answers to these questions will give administrators the necessary confidence to
%delegate their network's management through participatory networking.
