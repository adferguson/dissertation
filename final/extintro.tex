
\chapter{More detailed introduction}

\remark{
{\em Expected Outcome:} Reader understands the goals of participatory 
networking, and the division between \sys's static and dynamic components.
Reader understands the problem of protection and delegation, and why we
need the ShareTree
}

We begin our exploration of participatory networking by following a few
example interactions between the \sys controller and principals on the
network. These examples give a
flavor for how we use \sys in our environment, and introduce some of
the challenges involved in its implementation.
We then consider some of the additional considerations that influenced
\sys's design, which we will detail in the following chapters
(\S\ref{sec:ShareTree}-\ref{sec:FullSystem}).

\section{A Taste of \sys}

Principals interact with the \sys controller using a simple text-based protocol.
We start with the sequence of requests
necessary for an end-user, Alice, to reserve guaranteed bandwidth
for her flows. First, the network administrator must create a
\emph{share} for her flows which carries such a privilege: % adf-wt
%\begin{small}
\begin{verbatim}
NewShare aliceBW for (user=Alice)
  [reserve <= 10Mb] on rootShare.
\end{verbatim}
%\end{small}
Loosely speaking, a \sys share is a set of restricted network privileges
which can be exercised by specific users; we will refine this definition in
\xref{sec:ShareTree}.
{\color{red}
This share can be exercised on packets for which Alice is the sender
(
%\begin{small}
\verb/user=Alice/
%\end{small}
) } and allows the share's principals to reserve up to
10Mbps of guaranteed bandwidth, and does not authorize other requests.
Next, the administrator allows Alice to use this share:
%\begin{small}
\begin{verbatim}
Grant aliceBW to Alice.
\end{verbatim}
%\end{small}
Finally, Alice can request that her web traffic 
receive guaranteed bandwidth of 8Mbps for ten minutes, starting
twenty minutes in the future:
%\begin{small}
\begin{verbatim}
reserve(user=Alice,dstPort=80) = 8Mb
  on aliceBW from +20min to +30min.
\end{verbatim}
%\end{small}
{\color{red}
While
%\begin{small}
\verb/aliceBW/
%\end{small}
currently permits Alice to reserve bandwidth indefinitely,
a token bucket can be applied to limit
Alice's reservations (\xref{sec:gmb-requests}).
}
If Alice attempted to make a second reservation during an overlapping
time period (\eg, \verb/+25min to +35min/), the \sys controller would
reject her request because the combined reservations would have exceeded
the limit of 10Mbps set on her share.

\vskip 0.5em

In this second example,
a network administrator can create shares to delegate access control.
The following \sys request creates a share for Bob, who is using
the IP address 10.0.0.2, which allows him to deny traffic to his computer,
and which cannot be used for other requests:
%\begin{small}
\begin{verbatim}
NewShare bobAC for (dstHost=10.0.0.2)
  [deny = True] on rootShare.
Grant bobAC to Bob.
\end{verbatim}
%\end{small}
Using this share, Bob can, for example, block traffic from
Eve's host (10.0.0.3) for the next five minutes:
%\begin{small}
\begin{verbatim}
deny(dstHost=10.0.0.2, srcHost=10.0.0.3)
  on bobAC from now to +5min.
\end{verbatim}
%\end{small}
{\color{blue}
If Bob attempts to block traffic to Charlie's computer (10.0.0.4)
with this command:
%\begin{small}
\begin{verbatim}
deny(dstHost=10.0.0.4, srcHost=10.0.0.3)
  on bobAC.
\end{verbatim}
%\end{small}
his request will be rejected because its flowgroup is
not a subset of the flowgroup in
%\begin{small}
\verb/bobAC/
%\end{small}
. If Bob attempts to block all traffic from Eve by not including the
%\begin{small}
\verb/dstHost/
%\end{small}
restriction, his request will also be rejected.
}

\vskip 0.5em

While basic, these examples illustrate the standard workflow for
using \sys. In each case, a user with existing privileges delegated a
restricted portion of those privileges to another principal. Subsequently,
the second principal was able to invoke the capability dynamically.

These examples also show the two checks performed by the \sys controller
when a capability is invoked. First, \sys determines whether
the principal has been granted the appropriate privileges. Second, it
determines whether previously accepted requests preclude the new
request from being granted. For the first check, the \sys controller
maintains a data structure called the \emph{Share Tree} (\xref{sec:ShareTree});
for the second, a sequence of data structures called the \emph{Policy Trees}
(\xref{sec:PolicyTrees}). \sys maintains a Policy Tree for each event time point.

As more realistic examples, we have adapted several real applications to use
\sys. For example, our version of Hadoop uses \sys to create bandwidth
reservations during the shuffle phase and while replicating data in HDFS,
and our version of SSHGuard uses \sys to block malicious
traffic in the network, before it reaches the host's access link.
We further describe and evaluate these implementations
in \xref{sec:ApplicationUsage}.


\section{The Place for \sys}

Participatory networking is designed for networks within a single administrative
domain including corporate WANs, datacenters, campus or enterprise networks,
and home networks. In such networks, there is, logically, a single owner from
which authority can be delegated. \sys's design does not rely on changes
to the end-hosts, such as the use of a particular hypervisor, making it suitable
for networks with user-owned or managed devices.
While the principals in \sys are authenticated, they do not need to be trusted
as privileges are restricted by the system's semantics. We recognize, however,
that it may be possible to exploit combinations of privileges in an untoward fashion,
and leave such investigation and prevention as future work.

A participatory network is backwards-compatible with exiting networked
applications -- principals submit requests only to receive \emph{predictable}
behavior from the network. Unmodified applications continue to receive the
best-effort performance of existing packet networks.
While \sys requires application developers to modify their programs
to receive its benefits, participatory networking is not designed for
any particular application. The primitives, programmability and visibility
into network state which it provides make it a good substrate for developing 
systems with a greater focus on particular applications or scenarios, as
we describe in \xref{sec:building-block}.

As we saw in the examples above, our initial design implements a first
come-first serve service model, where the controller accepts or rejects requests
in a single-threaded fashion. This model is simple to understand and implement,
and treats all principals equally, as they are assured that
accepted requests will be implemented (subject to hardware failures, as we
discuss in \xref{sec:dynamics}). Although we have not
implemented them, participatory networks could support priorities across
principals. We describe how to implement priorities and their potential
ramifications in \xref{sec:conflict-resolution-operators}.

\subsection{SHOULD WE {\color{red} NAME} THIS SUB-SUB-SECTION?}

By default, requests in \sys never expire. This enables support for
fixed rules, such as a ban on TCP port 23 (telnet) traffic.
We envision \sys being used to establish
such rules, rather than a separate SDN application, so \sys can evaluate
dynamic requests within the context of both the fixed rules and
previous dynamic requests. This combination allows the \sys controller to detect,
resolve, and report conflicts between all rules in the network.
We expand upon \sys's position in the wider SDN ecosystem in \xref{sec:discussion}.

Because network clients may leave the network without notifying the controller,
there is a potential for accepted requests to become stale if a principal
leaves the network. In order to prevent the build-up of stale requests, \sys
can restrict principals from making perpetual requests, requiring them to
periodically renew their requests. In this configuration, resource requests
are similar to leases.

\sys offers two approaches for restricting the time length of requests.
The first is basic time limits -- requests made under such limits may not
last longer than a fixed number of seconds. These limits
may be appropriate for access control-type requests, for example.

The second approach is to restrict a principal's capabilities using a
token bucket-type model, similar to rate-limiting token buckets used in {\color{red} XXX}.
This approach is appropriate for privileges which request resources
shared across many principals, such as for guaranteed bandwidth, as
we detail in \xref{sec:gmb-requests}. When restricted by a token bucket,
principals must balance the ``cost'' of their requests against the ``budget''
provided by their token bucket; we leave the development of future
models to restrict requests as future work.
