% !TEX root = ferguson-dissertation.tex

\prefacesection{Acknowledgements}

{\raggedleft{}
\emph{``The pleasure we found in working together made us exceptionally patient; \\
it is much easier to strive for perfection when you are never bored.''}\\
\hfill Daniel Khaneman
}

\ \\
\ \\

This dissertation is the result of five years of wonderful collaboration.
The development and implementation of the ideas between these pages simply
would not have been possible without the hard work, deep discussions, and
shared excitement of all my co-authors: Rodrigo Fonseca, Arjun Guha, Betsy Hilliard,
Shriram Krishnamurthi, Chen Liang, Tim Nelson, Jordan Place, and Michael
Scheer. I owe them all a tremendous debt.

The success of these projects is also due to the technical and administrative
staffs of the Brown Computer Science department, particularly Lauren
Clarke, Jeff Coady, Mark Dieterich, Kathy Kirman, Dawn Reed, and Max Salvas.
With our interest in experimental infrastrucutre, I suspect systems researchers
pose a unique and difficult challenge for their departments' technical staff;
nonetheless, Jeff and Max were always ready to satisfy my creative requests,
and Mark at least feigned understanding when I brought down the department
routers in the middle of the night. I truly appreciate all of their help.

A very special and important thank you goes to my labmates:
Chen Liang, Jon Mace, Marcelo Martins, Jeff Rasley, Matheus Santos, Da Yu,
and Ray Zhou. Thanks to them and the life they brought to the lab, it was
okay to consider the systems lab ``home'' during the long stretches needed to
realize these projects. Thanks, you guys.

Shriram Krishnamurthi and his CSCI 1730 (Programming Languages) course
changed my life. I am surely not the first person to whom this has happened,
and I certainly won't be the last. By introducing me to the power of
type systems, functional programming, logic programming, and many other
topics, I finally had the tools to effect the changes I knew needed to
be made in today's networks. I have leaned heavily on his students over
the years: Spiros Eliopoulos, Arjun Guha, Ben Lerner, Tim Nelson, Joe Politz,
Justin Pombrio, and Hannah Quay-de la Vallee. For three years, I have always
sat within 15 feet of at least a few of them, and I am going to miss
terribly the luxury of easy access to their wisdom.

I have so much to thank Jennifer Rexford for. As an undergraduate, Jen's
introduction to computer networks was the only class in which I couldn't help
but start the homework just as soon as it was released; she had made everything
just too interesting! Later, she would write my letter of recommendation to
grad school, despite a) my last minute request, and b) being trapped in the
Bankok airport due to the December 2009 Thai protests. Since then, the instances 
of Jen's help to my development as a researcher have only continued to pile-up.

Critical amongst Jen's contributions was the early introduction to Frenetic.
Learning from, talking with, debating with, drinking with, and collaborating
with the wider Frenetic family --
Carolyn Anderson, Marco Canini, Spiros Eliopoulos, Nate Foster, Mike Freedman,
Arjun Guha, Rob Harrison, Nanxi Kang, Naga Katta, Chris Monsanto, Srinivas Narayana,
Mark Reitblatt, Josh Reich, Jennifer Rexford, Cole Schlesinger, Alec Story,
Dave Walker, and those to whom I apologize for forgetting
-- was truly instrumental to the success of these projects.
I couldn't have done this without their contributions: in papers, in conversation,
and on Github.

I also want to thank Peter Bodik and Srikanth Kandula for an incredibly maturing
internship at Microsoft Research. While the paper we produced on guaranteeing job
latency in data parallel clusters does not fit
within this dissertation, the experience taught me the level of work required for
these projects, and firmly connected to reality what I had read about Big Data
and warehouse-scale computing.

Finally, I must thank my advisor, Rodrigo Fonseca, who, from the first day,
gave me the freedom, support, and encouragement to pursue my intellectual
curiosities. Sometimes, this lead to failure ... but a few times, it worked out,
eventually bringing you, dear reader, this dissertation. From the first time
we wrote a paper together, I knew for certain I was lucky to have Rodrigo
as an advisor. I still am.
